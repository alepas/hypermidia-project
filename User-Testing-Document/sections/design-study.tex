\subsection{User profile definition}
We have defined 2 segments of user profiles heterogeneous for what concern the gender and homogeneous for what concern 	other age, civil state and tech capabilities.
\begin{itemize}
		\item \textbf{User profile one}
			\begin{itemize}
				\item \emph{Age range}: between 18 to 35 years old
				\item \emph{Civil State}: single or with fiancee
				\item \emph{Technology capabilities}: normal web user with no peculiar capabilities
			\end{itemize}
		\item \textbf{User profile two}
			\begin{itemize}
				\item \emph{Age range}: between 40 to 60 years old
				\item \emph{Civil State}: merried
				\item \emph{Technology capabilities}: normal web user with no peculiar capabilities
			\end{itemize}
\end{itemize}


\subsection{Scenarios}
The User testing is based on 3 scenarios
\begin{itemize}
		\item \textbf{Scenario 1}\\ You are planning to go to Valtellina in September and you are looking for an event related to 				wild animals; once you get it you wanna know more and decide to call the event organizer.
			\begin{enumerate}
				\item Surf the events and find the Septemeber ones
				\item Select an event that you like
				\item Find even organizer phone number to contact him/her and get more info
			\end{enumerate}
		\item \textbf{Scenario 2}\\ You would like to become volunteer of an association that protects wild animals.
			\begin{enumerate}
				\item Get info about the association
				\item Try to answer to your question reading faqs
				\item Send a request to become volunteer
			\end{enumerate}
		\item \textbf{Scenario 3}\\ You are discovering Wild Care services and you would like to get more information about the 			event related to a certain service that you liked the most.
			\begin{enumerate}
				\item Surf the services
				\item Select a service that you find interesting 
				\item Pick a related event and see its details
			\end{enumerate}
\end{itemize}

\subsection{Variables to measure}
To evaluate the task execution we have chosen to adopt the following metrics:
\begin{itemize}
	\item \emph{time of execution}: the clock starts when the user directs his/her attention to the application
	\item \emph{success rate}: to a "complete success" is assigned a value equal to 1, to a "partial success" is assigned a value 		equal to 0.5 and to a "failure" a value equal to 0.0
	\item \emph{perceived difficulty}: an oral evaluation between 0 and 5 given imemediately after the task execution
	\item \emph{errors}: integer that express how many "wrong" links have been clicked to reach the goal or wrong paths have 		been taken
	\item \emph{satisfaction}: an oral evaluation between 0 and 5 given imemediately after the task execution
\end{itemize}

\subsection{Final survey}
After all tasks execution, every user fulfilled a questionnaire formed by N questions with a rating between 0 to 5. We have used Forms Pro software to create the survey and collect data; we have reported here all questions for completeness.
\begin{enumerate}
	\item How much useful did you find the topbar? 
	\item How much easy has it been to find events for month?
	\item How much easy has it been to find services related to events?
	\item How much easy has it been to find events organized by a volunteer?
	\item How much easy has it been to find volunteer's details?
	\item How easily did you find the mission of the association?
	\item Do you find the text layout readable?
	\item Do you find images dimension good?
	\item Do you find website layout consistent?
	\item Are semantic close events also close into the space?
\end{enumerate}
The survey can be found at link \url{https://forms.office.com/FormsPro/Pages/ResponsePage.aspx?id=8eAiizYZfk-CTpheWGbGkZgSwAPI3GJPvXJ0APwN3yxUNU42NjFRRTIwTlA5SENQWTNKVEwwQzRBTS4u}.
